\documentclass{jsarticle}
\usepackage[dvipdfmx]{graphicx}
\usepackage{tikz}

\usepackage{ascmac}
\usepackage{amsmath}
\usepackage{amsthm}
\usepackage{amssymb}
\usepackage{amsfonts}
\usepackage{latexsym}
\usepackage{mathtools}
\usepackage{float}
\usepackage{url}
\renewcommand{\qed}{\unskip\nobreak\quad\qedsymbol}
\usepackage{endnotes}
\renewcommand{\notesname}{注}
\let\footnote=\endnote
\renewcommand{\theendnote}{\arabic{endnote})}
\usepackage{etoolbox}
\patchcmd{\enoteformat}{1.8em}{0pt}{}{}



\begin{document}
\begin{center}
{\large \textbf{モダンの誕生と経過}}\par
\vspace{0.5em}
荒木 理求\par
\texttt{rikuman81129@gmail.com}\par
2025年8月10日
\end{center}

\section*{はじめに}
橋爪大三郎は、構造主義を近代主義の成果として評価しながらも、反モダニズムたるポストモダン思想の一環に位置付ける\cite{Hashidume1988}[p.226]。しかし構造主義を成熟したモダニズムの到達点と見做すのであれば、むしろモダニズムに肯定的な方法論を提供したという解釈も可能なはずだ。\\
\indent この逆説的なアイディアは魅力的だが、モダンとポストモダンに明確な区別を与えない以上、生産的な議論はできない。そこで本稿では、ポストモダニズムを理解するための下ごしらえとして\footnote{『孫氏』には「彼を知り己を知れば百戦殆からず」とある。筆者にとっての敵は、何を隠そう、ポストモダニズム(の一部、行き過ぎたもの)である。}、モダニズムの誕生から変容の時代を概観し、モダニズムに対する一つの態度を提示する。\\
\indent 具体的には西洋を舞台に、モダニズムの源流を遠近法の発明に求める。すなわち、ルネサンス期に成立した〈視点〉の制度化$=$〈自己〉\footnote{当然、ルネサンス以前も「自己」という概念は存在していた。ここでは以下に述べるような意味での「近代的な自己」を〈自己〉と表す。しばしばその意味を強調するため、近代的(な)〈自己〉とも書く。}の誕生を出発点に据え、その〈自己〉が文化、言語、身体にまで浸透・拡張し、さらに相対化される過程をモダニズムと定義する。\\


\section*{モダニズムの起源}
\indent 遠近法の登場と〈自己〉の誕生を対応付けるのであれば、ロットマンが\cite{Rotman1987}で指摘する通り、``0''、消失点、そして想像上の通貨の関係について言及すべきだろう。\\
\indent``0''は約1300年前、インドでヒンドゥー数字の一つとして成立し、アラブ商人によって地中海世界に広まり、10世紀にはアラブ文化圏で広く使用されていた。しかしキリスト教世界のヨーロッパでは、神の創造に矛盾する「無」を意味するために、10~13世紀の間は受け入れられなかった。そのため、14世紀に北イタリアで窯業資本主義が台頭し、数字の使用主体が聖職者から商人や建築家に変わることで初めて``0''が受容されるに至った。意外にも、ヒンドゥー数字が支配的となったのは17世紀初頭のことである。\\
\indent さてロットマンは ``0''が体系内部の記号、すなわち他の数字同様に演算の対象となる一方、「数の不在を表す」メタ記号(meta-sign)としても機能する二面性に注目する。メタ記号とは記号体系に外部から秩序を与える記号であり、この場合``0''は数え始めとなる基準点、いわば「数える主体」の痕跡とも解釈できる。したがって``0''というメタ記号を理解することは、「数える主体」を知覚する新たな主体、メタ主体(meta-subject)の誕生を要請する。\\
\indent 遠近法においても、画面内の線が集まる消失点は、画中の特定の点であると同時に「他の記号がない」ことを表すメタ記号として機能している。\\
\indent 中世ヨーロッパの宗教画では、宗教的価値に忠実であることが重視され、多くの場合写実性は二の次であった。しかし1425年、建築家のブルネレスキが鏡を用いた実験から遠近法を発明し、1436年にはアルベルティが『絵画論』として理論化、さらにピエロ・デラ・フランチェスカらによって洗練された。消失点とは画家の視点そのものであり、鑑賞者に画家の視点に立つことを強制するので、「視る主体」に意識的なメタ主体の誕生を促す。このブルネレスキらの仕事は〈視点〉の制度化と呼べる。遠近法の誕生$=$〈視点〉の制度化は、ルネサンス期の人文主義に基づく人間中心世界像の登場と重なり、近代的意味での〈自己〉の誕生と見做すことができる。この〈視点〉の制度化は、モダニズムの時代に芸術家が自らのアイデンティティを問う土台となった。\\
\indent1637年、デカルトが『方法序説』で平面座標の概念を確執し、幾何学と代数学を結び付けた。直行する直線の交点(原点)と$(0,0)$の対応は、消失点$=$〈視点〉と$(0,0)$の対応と似た性質を持つ。例えば18~19世紀に形作られた曲線論・曲面論では、各点ごとに3次元空間の直交座標系を与えて計算するが、これは〈視点〉の移動に対応して$(0,0,0)$も移動するように見える。ヴァイルは\cite{Weyl1949}[p.75]にて、原点とは観測主体の立ち位置$=$〈視点〉を表しながらもその主体自体は描かれない「自己消滅の必然的残滓」(necessary residue of ego-extinction)であると表現する。このように``0''と消失点には、メタ記号であること以上の共通点が見られる。\\
\indent 想像上の通貨もまたメタ記号と見做せる。中世の金貨は素材自体に価値があるためアイコニックな記号だが、ルネサンス期には国際商業の発展で金貨の品質差(agio)が問題化し、ヴェネツィアやアムステルダムの銀行が金や銀と交換可能な想像上の通貨として約束手形を発行した。これは金としての物理的な形を持たないと同時に、通貨体系を外部から成立させるメタ記号である。のちに紙幣として流通するが、ここでもそれを使用する主体としてメタ主体が必要になる。\\
\indent ここまで、``0''、消失点、想像上の通貨のようなメタ記号によって、それを扱うための、主体の存在に自覚的なメタ主体が誕生することを見た。前述の通り、以下では特に遠近法による〈視点〉の制度化を、近代的な〈自己〉の起点と考える。

\section*{近代的〈自己〉の確立と分散・相対化}
中世においては神との関係の中に自己が存在し、自由意志は限定的であったが、ルネサンスに差し掛かると、キリスト教的信仰と人間理性への期待の中に遠近法が誕生し、近代的な〈自己〉の萌芽が感じられるのであった。しかしこの〈自己〉を認識するメタ主体の誕生は、ナルシスが水面に映る自分の鏡像を他者として了解する段階に対応しており、〈自己〉への意識が生まれる前の「絶対的な自分」との離別、すなわち〈自己〉の分散・相対化を意味する。柴田元幸は、この〈神の退席〉の進行によってアメリカン・ルネサンスの作家たちが「自分が何者なのか」について本当に苦しんだことを指摘し、メルヴィル『白鯨』におけるナルシスの物語への言及をその表出として引用する\cite{Shibata2017}[pp.5-6\&215-216]。\\
\indent ここでは〈神の退席〉と〈自己〉の分散・相対化のプロセスを、1920年代のアメリカを具体例として観察する。\\
\subsection*{キリスト教世界からの解放}
1920年代のアメリカは、第一次世界大戦後の好況と技術革新を背景に、生活様式と価値観が劇的に変化した時代だった。戦争体験を通じて従来の道徳規範に対する信頼が失われ、「快楽を今この場で享受する」ことが行動原理として台頭したことが大きい。海外勤務や戦地での経験は、とりわけ性的な規範の弛緩に寄与し、戦後の世代では婚前・婚外関係への忌避感が希薄になった。このような背景のもと、家電製品の普及や商業サービスの発達により家庭内労働が軽減され、女性の社会進出も著しく進展した。家庭はもはや聖域ではなく、都市部では単なる生活拠点としての機能を担うに過ぎなくなった。\\
\indent また、フロイトの精神分析が一般にも流布し、「性」を中心とした人間理解が正当化されるようになったことで、「自己探求」は「性的自己肯定」とほとんど同義となり、リビドーやエディプス・コンプレックスといった専門用語も一般の語彙に組み込まれていった。これに加えてアメリカ固有の要因、例えば禁酒法の存在、自動車の普及、性や告白をテーマとした雑誌や映画の爆発的な拡大が相乗的に作用し、従来の性道徳やジェンダー規範は目に見えて揺らいでいくこととなる。\\
\indent 禁酒法は本来、キリスト教的な節度と公共道徳の回復を志向した制度であったが、実際にはその矛盾が新たな享楽空間を生み出す結果となった。スピークイージーと呼ばれる地下酒場では、男女が共に酒を酌み交わす光景が日常となり、従来のジェンダー規範や家庭的モラルは形骸化していった。アルコールという媒介を通して、異性との接触や私的空間での自由なふるまいが正当化され、性的自己決定を可能にする新しい場が社会的に容認されるようになっていった。\\
\indent こうした変化の中でも特に顕著だったのが、女性の外見とライフスタイルの劇的変容である。スカート丈はわずか数年で足首から膝上へと変化し、ボブカットが全世代的に流行した。化粧は恥じらいの対象ではなくなり、美容院や化粧品産業の発展とともに、化粧を通じた〈自己〉の演出が「日常」となった。性的魅力と自由を両立させようとする女性像は、もはや「母性」ではなく、「恋人」や「遊び相手」としての役割を自己規定の軸に据えるようになっていた。\\
\indent この変化はマナーや道徳の領域にも波及し、喫煙や飲酒のジェンダー的境界が解体されるなど、従来の価値体系は次第に効力を失っていった。こうして1920年代のアメリカ社会は、「自由であること」と「性的に魅力的であること」を同時に肯定する〈自己〉を生成しつつ、キリスト教的価値観に基づく人間像からの決定的な離脱を果たしたのである。\\

\subsection*{基盤を失った〈自己〉の瓦解}
しかし超越的原理なしに〈自己〉を確立するのは難しい。実際、キリスト教的な道徳が崩れ去ると、それに取って代わる絶対的な価値体系が見つかることはなかった。〈自己〉の発見やそれに続くキリスト教的世界からの解放の契機となった、ある意味反宗教的な人文主義は、逆説的にはキリスト教に依存したものであり、近代的な〈自己〉を形成するには役立たないからだ。したがって、基盤のない「自由」を標榜するこの時代のアメリカは、混迷を極めた。\\
\indent 社会には性への過剰な執着(obsession)が広がり、文学、新聞、雑誌、映画においても性的テーマが氾濫した。ピューリタン的な慎みが失われたことで、性的な話題は日常会話の中心になり、「自由」であることは性について語ることであるかのような空気が漂った。さらにモラルの崩壊とともに日常のマナーも崩壊し、デカダンスの空気が社会全体に広がった。食事会での無礼、遅刻、家具を壊す乱痴気騒ぎ、節度のない飲酒などが横行し、人々は「自由」を履き違え始めた。\\
\indent 夫婦関係においても、自由恋愛や浮気が正当化されたが、実際には多くの人がその「自由」を受け入れきれずに苦しんだ。例として、ある女性は「夫の恋愛を許す」と言いつつも、実際に相手の女性を見ると身体的に気分が悪くなるほどの葛藤を抱えた。このように愛は「瞬間的な快楽」へと矮小化され、ロマンティック・ラブや人生の意味は失われていった。\\
\indent 多くの人々は慣れ親しんだ価値観を簡単に捨てられず、外面的には「自由」を受け入れつつも、感情現実との矛盾に思い悩んでいたのである。結果として、人々は「自由」を獲得した代償に、「私は誰なのか」という根源的な問いに直面することになる。\\
\indent 絶対的な価値を失った時代において、その空白を埋めるかのように登場したのが、資本主義と消費活動による〈自己〉の規定であった。\\

\subsection*{相対的な〈自己〉の誕生}
\indent 1920年代初頭、アメリカは第一次世界大戦後の不況を乗り越え、1923年以降、いわゆる「繁栄の高原」(The plateau of prosperity)と呼ばれる持続的経済成長の時代に突入する。ハーディングの死後大統領に就任したクーリッジの時代には、経済指標が安定的に上昇するなかで、人々の信頼は宗教や政治から離れ、「アメリカのビジネス」そのものへと向けられていった。\\
\indent この時期、人々はビジネスマンにこそ時代の指導者を見出し、かつて聖職者や哲学者に託していた社会規範の生成を、企業家や経済的成功者に託すようになる。成功が道徳となり、富が人格の証明とされる構図が生まれた。“The business of America is business”という言葉に象徴されるように、資本主義の原理は経済活動の枠を超え、人生の指針そのものへと変貌を遂げたのである。\\
\indent このような価値観の変化と並行して、社会的階層の構造もまた大きく再編された。従来、労働者階級を支えていたプロレタリア意識や階級的団結は次第に弱まり、代わって消費を通じた〈自己〉形成が人々の間に広がりつつあった。フォードの自動車産業の展開はこの変化を如実に示している。1927年、フォードは庶民に普及したT型フォードの生産を停止し、新型のモデルAを発表した。これは単なる商品刷新にとどまらず、「安価で実用的な製品」から「美しさ・速さ・スタイルを備えた製品」への価値観の転換を示している。\\
\indent モデルAの登場には、全国的な熱狂が伴った。新車の発表に100万人が殺到し、デトロイトのショールームには10万人が詰めかけた。車の色や形、速度そのものが欲望の対象となり、自動車は単なる移動手段から「〈自己〉の拡張としてのシンボル」へと変化した。ここではもはや「所有」ではなく、「何を、なぜ選ぶか」が重要であり、消費行為そのものが社会的アイデンティティの表現方法となっている。\\
\indent この熱狂の背景には、1920年代後半にアメリカ社会を席巻した“Ballyhoo”、すなわち派手な宣伝とごまかしの風潮があった。当時は、政治・宗教・ビジネス・芸能といったあらゆる領域が実態よりも話題性や演出性を競う時代となっており、何がどれだけ注目されたかが、そのまま価値の指標とされた。新聞・雑誌・ラジオといったメディアは、商品の機能を伝えること以上に人々の感情を揺さぶるストーリーや印象操作に傾斜し、粗末な商品すらも「買うべきもの」に仕立て上げていった。
こうした風潮のなかで、広告産業は単なる商品告知の手段から、〈自己〉を形成する文化装置へと変貌を遂げていく。広告はもはや商品の性能を説明するのではなく、若さ・魅力・富・他者からの羨望への欲望そのものを刺激する装置へと変化したのである。特に化粧品や生理用品のような、これまで公然と語られなかった商品でさえ、率直さや現代性を象徴する語り口によって、消費者の羞恥心や不安を巧みに刺激し、購買を促す物語が量産された。\\
\indent 典型的な事例が、口臭対策薬「リステリン」の広告である。1920年代にこの製品は、ハリトーシス(halitosis)という医学的根拠の乏しい造語を用いて、「何度も花嫁介添人を務めたが、花嫁にはなれない」「彼女が去った理由はこれだ」など、孤独や結婚失敗といった人生の危機を製品と結びつける物語を展開した。広告に登場する人物たちは、「欠落感を抱く人間」の象徴としてのフィクションであり、その欠如を補う手段として商品が提示される。このように購買は単なる選択ではなく救済となり、消費行動が〈自己〉を規定するものとして正当化されていった。\\
\indent ここにおいて、購買行為は「必要だから買う」という合理的判断ではなく、「欲望を喚起され、物語に巻き込まれて買わされる」という演出されたプロセスへと変化したことがわかる。しかもこの演出は、単に商品を買わせるのではなく、「何を買うか」を通じて「私は誰なのか」を提示させる。すなわち、商品は〈自己〉を外部から構成するコードとして機能するようになった。「必要だから買う」から「買った結果必要だったことになる\footnote{もちろん衝動買いを後悔する場合も多いが、ここでは需要よりも先に供給があることが重要である。また、例えば連絡手段として入手したはずのスマートフォンによって、むしろライフスタイルが変化することを考えると、この逆転も理解しやすいだろう。}」への変容、より広くは「本来人間の道具に過ぎないはずのモノが、むしろ人間を規定する」という逆転は、マクルーハンの“The medium is the message”という主張の一例にも思える\cite{McLuhan1964}。\\
\indent さて、こうした消費文化の土壌を支えたのが、フランスで発明されたデパートという空間である。\\
\indent 19世紀前半までのフランスの商店では、入店および退店の自由がないうえに、商品には値段がついておらず、値段交渉は熾烈を極めた。そのため一部の上流階級を除いて、買い物とは「必要を満たす」ための行為であった。しかしウィンドー・ショッピングの快楽を生んだパレ・ロワイヤルやパサージュ、またいくらか規制を緩め、圏外からの来店を狙ったマガザン・ド・ヌヴォテ(流行品店)に学んだブシコ―が経営を始めたボン・マルシェは、ディスプレイに創意工夫を凝らしたことで欲望換気装置として機能し、またアッパー・ミドルのライフスタイルを提唱した。ブシコ―の演出は中産階級の消費への欲望を目覚めさせ、消費願望のあとに実際の消費がくる、消費資本主義の構造を生み出したのである。鹿島茂は「デパートを発明したブシコ―こそ資本主義の発明者である」とまで主張する\cite{Kashima2023}p.16。\\
\indent 上に述べた変化は、消費者を取り巻く環境にも現れている。1920年代中盤には、ラジオ、冷蔵庫、化粧品、電気製品といった新興商品が爆発的に普及し、人々の生活様式は急速に現代化した。都市生活に象徴される服装や趣味は田舎にも波及し、農村の若者が都市の流行をなぞるような現象が各地で観察された。人々は階級的連帯ではなく、同じ化粧品や衣服、家具を通じて同時代性を共有し始めたのである。\\
\indent かくして、職業や階級に代わり、消費活動が〈自己〉を定義するようになった。1924年には、ヘンリー・フォードを大統領に推す運動が起こるなど、経済的成功が政治的信頼を意味する状況が生まれ、「良き市民」とは「経済的な成功者」であるという価値観が浸透していった。\\
\indent 特筆すべきは、これらの変化が上層だけでなく、労働者層にも及んでいたという点である。中古のビュイック車を所有し、富裕な政治家に喝采を送る労働者たちは、もはや自らを「抑圧された存在」ではなく、「消費を通じて自由を選び取る存在」として認識していた。階級意識の解体は、資本主義の倫理に基づいた新たな〈自己〉理解へとつながっていたのである。信仰や出自・血縁を中心に据える絶対的な〈自己〉から一転、資本主義を基盤とした相対的な〈自己〉が確立されたと言えよう。

\section*{むすびにかえて}
\indent 以上が消費社会誕生の経緯であるが、1世紀前のことにもかかわらず、我々の生きる現代とさして変わりないように思える。実際、今日の中高生は、YouTubeで登録しているチャンネルによって〈自己〉を主張する\footnote{ここにおける「消費」とは、動画の再生である。ちなみに、「メンバーシップ」という有料会員の制度もある。}といった具合だ。

\theendnotes
\begin{thebibliography}{99}
\bibitem{Kashima2023} 鹿島 茂.
\newblock 『デパートの誕生』
\newblock 講談社学術文庫, 2023.

\bibitem{Shibata2017} 柴田 元幸.
\newblock 『アメリカン・ナルシス メルヴィルからミルハウザーまで』
\newblock 東京大学出版会, 2017.

\bibitem{Hashidume1988} 橋爪 大三郎.
\newblock 『はじめての構造主義』
\newblock 講談社現代新書, 1988.

\bibitem{Allen1931} Allen, Frederick Lewis.
\newblock {\em Only Yesterday: An Informal History of the 1920's}.
\newblock Harper\&Brothers, 1931.

\bibitem{McLuhan1964} McLuhan, Marshall.
\newblock {\em Understanding Media: The Extensions of Man}.
\newblock McGraw-Hill, 1964.

\bibitem{Rotman1987} Rotman, Brian.
\newblock {\em Signifying Nothing: The Semiotics of Zero}.
\newblock Stanford University Press, 1987.

\bibitem{Weyl1949} Weyl, Hermann.
\newblock {\em Philosophy of Mathematics and Natural Science}.
\newblock Princeton University Press, 1949.
\end{thebibliography}
「モダニズムの起源」は\cite{Rotman1987}、「近代的〈自己〉の確立と分散・相対化」は\cite{Allen1931}を大いに参考にした。

\end{document}
