\documentclass{jsarticle}
\usepackage[dvipdfmx]{graphicx}
\usepackage{tikz}
\usepackage[linguistics]{forest}
\usepackage{ascmac}
\usepackage{amsmath}
\usepackage{amssymb}
\usepackage{amsfonts}
\usepackage{latexsym}
\usepackage{mathtools}
\usepackage{float}
\title{Poole(2011)$\colon$ Chapter7 Further Exercises}

\author{荒木 理求\\23120086@ed.tmu.ac.jp}
\date{\today}
\begin{document}
\maketitle

\section{Exercise 1}
\begin{boxnote}
以下の文のLF表示をかけ{$\colon$}\\
\begin{enumerate}
\item[]
\vspace{-10pt}
\indent 1. Everyone left yesterday.\\
\indent 2. John said that Bill believed some politician.\\
\indent 3. Zhangsan xiangzin [shei mai-le shu]?[Chinese]\\
\hspace{12pt}Zhangsan believe who bought book\\
\hspace{12pt}'Who does Zhangsan believe bought books?'
\end{enumerate}
\end{boxnote}
\vspace{10pt}
\subsection{}
\begin{enumerate}

%%問1.1
\item[(1)]
S構造$\colon$
[{\tiny CP} [{\tiny IP} Everyone left yesterday]].
\item[(2)]
LF表示$\colon$
[{\tiny CP} [{\tiny IPA} Everyone$_i$ [{\tiny IPB} $t_i$ left yesterday]]]
\vspace{10pt}
\\
(2)
{\fontsize{9pt}{9pt}\selectfont
\begin{forest}
 %% triangle
 delay={where n children=0{if={instr("P",content("!u"))}{roof}{}}{}}
[IP$_A$[DP[Everyone$_i$ ,name=spec IPA]][IP$_B$[DP[$t_i$ ,name=spec IPB]][I$^{\prime}$[I[\lbrack Past\rbrack]][VP[V$^{\prime}$[V$^{\prime}$[V[left]]][AdvP[yesterday]]]]]]]
 %%movement
\draw[->](spec IPB)to[out=south west ,in=south](spec IPA);
\end{forest}}
\end{enumerate}

%%問1.2
\newpage
\subsection{}
\begin{enumerate}

\item[(3)]
S構造$\colon$
[{\tiny CP} [{\tiny IP} John said [{\tiny CP} that [{\tiny IP} Bill believed some politician]]]].
\item[(4)]
LF表示$\colon$
[{\tiny CP} [{\tiny IP} John said [{\tiny CP} that [{\tiny IPA} some politician$_i$ [{\tiny IPB} Bill believed $t_i$]]]]].
\vspace{10pt}
\\
(4)
{\fontsize{9pt}{9pt}\selectfont
\begin{forest}
 %% triangle
 delay={where n children=0{if={instr("P",content("!u"))}{roof}{}}{}}
[CP[C$^{\prime}$[C[that]][IP$_A$[DP[every politician$_i$ ,name=spec IPA]][IP$_B$[DP[Bill]][I$^{\prime}$[I[\lbrack Past\rbrack]][VP[V$^{\prime}$[V[believed]][DP[$t_i$ ,name=object]]]]]]]]]
 %%movement
\draw[->](object)to[out=south west ,in=south](spec IPA);
\end{forest}}
\end{enumerate}
\vspace{-20pt}
\subsection{}
\begin{enumerate}

\item[(5)]
S構造$\colon$
[{\tiny CP} [{\tiny IP} Zhangsan xiangzin [{\tiny CP} [{\tiny IP} shei$_i$ mai-le shu]]]]?
\item[(6)]
LF表示$\colon$
[{\tiny CP} shei$_i$ [{\tiny IP} Zhangsan xiangzin [{\tiny CP} $t^{\prime}_i$ [{\tiny IP} $t_i$ mai-le shu]]]]
\vspace{10pt}
\\
(6)
{\fontsize{9pt}{9pt}\selectfont
\begin{forest}
 %% triangle
 delay={where n children=0{if={instr("P",content("!u"))}{roof}{}}{}}
[VP[V$^{\prime}$[V[xiangzin]][CP[DP[shei$_i$ ,name=spec CP2]][C$^{\prime}$[C[$\emptyset$]][IP[DP[$t_i$ ,name=spec IP]][I$^{\prime}$[I[\lbrack Past\rbrack]][VP[mai-le shun]]]]]]]]
 %%movement
\draw[->](spec IP)to[out=south west ,in=south](spec CP2);
\end{forest}}
{\fontsize{7pt}{7pt}\selectfont
\begin{forest}
 %% triangle
 delay={where n children=0{if={instr("P",content("!u"))}{roof}{}}{}}
[CP[DP[shei$_i$ ,name=spec CP1]][C$^{\prime}$[C[$\emptyset$]][IP[DP[Zhangsan]][I$^{\prime}$[I[\lbrack Pres\rbrack]][VP[V$^{\prime}$[V[xiangzin]][CP[DP[$t^{\prime}_i$ ,name=spec CP2]][C$^{\prime}$[C[$\emptyset$]][IP[$t_i$ mai-le shun]]]]]]]]]]
 %%movement
\draw[->](spec CP2)to[out=south west ,in=south](spec CP1); 
\end{forest}}


\end{enumerate}

\section{Exercise 2}
\begin{boxnote}
不定冠詞の$\mathit{a(n)}$は$\mathit{some}$と同様に, 存在量化詞(existential quantifier)として機能する. 作用域の曖昧性(scope ambiguity)の議論を用いて, これを示せ\footnote{pp.207-11とあるのは誤植で, 正しくはpp.204-206 Scope ambiguity and QRを参照せよ}.
\end{boxnote}
\vspace{10pt}
\noindent まず次の(1)を考える$\colon$
\begin{enumerate}
\item[(1)]
Every student likes a teacher.
\item[(2)]
S構造$\colon$
[{\tiny CP} [{\tiny IP} every student [{\tiny VP} likes a teacher]]].
\end{enumerate}
この解釈は以下の2通りである$\colon$
\begin{enumerate}
\item[(3a)]
$\forall$ $x$$\in$$\{$生徒その1, 生徒その2, $\cdots$$\}$$\exists$ $y$$\in$$\{$教員その1, 教員その2, $\cdots$$\}$($x$は$y$が好き)
\item[(4a)]
$\exists$ $y$$\in$$\{$教員その1, 教員その2, $\cdots$$\}$$\forall$ $x$$\in$$\{$生徒その1, 生徒その2, $\cdots$$\}$($x$は$y$が好き)
\end{enumerate}
(3a)は「生徒全員に好きな教員が存在する」, (4a)は「生徒全員に好かれている教員が存在する」ことに他ならず, LFはそれぞれ次の様になる$\colon$
\begin{enumerate}
\item[(3b)]
[{\tiny CP} [{\tiny IPA} every student$_i$ [{\tiny IPB} a teacher$_j$ [{\tiny IPC} $t_i$ [{\tiny VP} likes $t_j$]]]]] (a teacherを先に繰り上げた場合)
\item[(4b)]
[{\tiny CP} [{\tiny IPA} a teacher$_j$ [{\tiny IPB} every student$_i$ [{\tiny IPC} $t_i$ [{\tiny VP} likes $t_j$]]]]] (every studentを先に繰り上げた場合)
\end{enumerate}
また$\mathit{a(n)}$を主語位置に置いた, 
\begin{enumerate}
\item[(5)]
A student likes every teacher. 
\end{enumerate}
においても2通りの解釈が可能で, QRの順番によって説明される. したがって$\mathit{a(n)}$は存在量化詞と言える.

\newpage
\section{Exercise 3}
\begin{boxnote}
以下の文を考えよ{$\colon$}
\begin{enumerate}
\item[]
\indent 1. Some student whispered that Bill read every book.\\
\indent 2. Some student believed that Bill read every book.
\end{enumerate}
$\mathit{some}$と$\mathit{every}$の相対的な作用域としてあり得るものは何か. また, このデータはこれまでの議論に矛盾するか.
\end{boxnote}
\vspace{10pt}
\noindent 1の解釈として,
\begin{enumerate}
\item[(1a)]
$\exists$ $x$$\in$$\{$生徒その1, 生徒その2, $\cdots$$\}$$\forall$ $y$$\in$$\{$本その1, 本その2, $\cdots$$\}$($x$はBillが$y$を読んだと噂した)
\end{enumerate}
があり得る. これは, 「ある生徒がBillは全ての本を読んだと噂した」ことを意味し, LFは次の様$\colon$
\begin{enumerate}
\item[(1b)]
[{\tiny IPA} some student$_i$ [{\tiny IPA} $t_i$ whispered that [{\tiny IPB} every book$_j$ [{\tiny IPB} Bill read $t_j$]]]].
\end{enumerate}
一方, 2の解釈として,
\begin{enumerate}
\item[(2a)]
$\exists$ $x$$\in$$\{$生徒その1, 生徒その2, $\cdots$$\}$$\forall$ $y$$\in$$\{$本その1, 本その2, $\cdots$$\}$($x$はBillが$y$を読んだと信じていた)
\item[(3a)]
$\forall$ $y$$\in$$\{$本その1, 本その2, $\cdots$$\}$$\exists$ $x$$\in$$\{$生徒その1, 生徒その2, $\cdots$$\}$($x$はBillが$y$を読んだと信じていた)
\end{enumerate}
があり得る. (2a)は「ある生徒がBillは全ての本を読んだと信じていた」, (3a)は「任意の本について, Billがその本を読んだと信じていた生徒が存在する」ことに他ならず, LFはそれぞれ次の様になる$\colon$
\begin{enumerate}
\item[(2b)]
[{\tiny IPA} some student$_i$ [{\tiny IPA} $t_i$ believed that [{\tiny IPB} every book$_j$ [{\tiny IPB} Bill read $t_j$]]]].
\item[(3b)]
[{\tiny IPA} every book$_j$ [{\tiny IPB} some student$_i$ [{\tiny IPC} $t_i$ believed that Bill read $t_j$]]].
\end{enumerate}
しかし, $\S$7.2 Quantifier raising and Subjacencyにおいて, 数量詞は最も近いIPに付加(adjoin)し, それ以上遠くに移動できないことを見た. したがって, ここまで扱ってきた規則では, (3b)の派生を得ることはできず, 矛盾が生じる.


\newpage
\section{Exercise 4}
\begin{boxnote}
代名詞が束縛変項(bound variable)として機能するためには, 演算子(operator)によってc統御されなければならないという仮定のもとで, 以下の文はQR(quantifier raising)の存在の証左となる. これを説明せよ.
\begin{enumerate}
\item[]
\indent 1. A report card about every student$_i$ was sent to his$_i$ parents.\\
\indent 2. *The woman who loved every man$_i$ decided to leave him$_i$.
\end{enumerate}
\end{boxnote}
%%問4.1
\vspace{10pt}
\subsection{}
\begin{enumerate}
\item[(1)]
S構造$\colon$
[{\tiny CP} [{\tiny IP} [{\tiny DP} a report card about every student$_i$]$_j$ was sent $t_j$ to his$_i$ parents]].
\vspace{10pt}
\\
(1)
{\fontsize{9pt}{9pt}\selectfont
\begin{forest}
 %% triangle
[PP[P$^{\prime}$[P[about]][DP[every student$_i$, roof]]]]
\end{forest}}
\end{enumerate}
\indent 1では, his$_i$が束縛変項として機能しているので, 演算子であるevery student$_i$にc統御される必要がある. しかし, S構造でDP-every student$_i$がc統御しているのは P-aboutのみである. そこでQRの存在を認めれば, LFは次の様になる$\colon$
\begin{enumerate}
\item[(2)]
LF表示$\colon$
[{\tiny CP} [{\tiny IPA} every student$_i$ [{\tiny IPA} [{\tiny DP} a report card about $t_i$]$_j$ was sent $t_j$ to his$_i$ parents]]].
\vspace{10pt}
\\
(2)
{\fontsize{9pt}{9pt}\selectfont
\begin{forest}
 %% triangle
delay={where n children=0{if={instr("P",content("!u"))}{roof}{}}{}}
[IP$_A$[DP[every student$_i$]][IP$_B$[DP[\lbrack a report card about $t_i$ \rbrack$_j$]][I$^{\prime}$[I[was]][VP[sent $t_j$ to his$_i$ parents]]]]]
\end{forest}}
\end{enumerate}
(2)ではたしかにevery student$_i$がhis$_i$をc統御しており, 問題の仮定を満たす.
\newpage
\subsection{}
\begin{enumerate}
\item[(3)]
S構造$\colon$
*[{\tiny CP} [{\tiny IP} [{\tiny DP} The woman [{\tiny CP} who$_j$ [{\tiny IP} $t_j$ loved every man$_i$]]] decided to leave him$_i$]].
\vspace{10pt}
\\
(3)
{\fontsize{9pt}{9pt}\selectfont
\begin{forest}
 %% triangle
delay={where n children=0{if={instr("P",content("!u"))}{roof}{}}{}}
[VP[V$^{\prime}$[V[loved]][DP[every man$_i$]]]]
\end{forest}}
\end{enumerate}
\indent2もevery man$_i$とhim$_i$が同一指標をもち, 束縛変項の解釈を仕向けているが, S構造においてDP-every man$_i$がc統御するのはV-lovedのみである. このときQRの存在を認めれば, LFは次の様になる$\colon$
\begin{enumerate}
\item[(4)]
LF表示$\colon$
[{\tiny CP} [{\tiny IP} The [{\tiny NP} woman [{\tiny CP} who$_j$ [{\tiny IPA} every man$_i$ [{\tiny IPB} $t_j$ loved $t_i$]]] decided to leave him$_i$]].
\vspace{10pt}
\\
(4)
{\fontsize{9pt}{9pt}\selectfont
\begin{forest}
 %% triangle
delay={where n children=0{if={instr("P",content("!u"))}{roof}{}}{}}
[NP[N$^{\prime}$[N[woman]][CP[DP[who$_j$]][C$^{\prime}$[C[$\emptyset$]][IP$_A$[DP[every man$_i$, name=spec IPB]][IP$_B$[DP[$t_j$]][I$^{\prime}$[I[\lbrack Past\rbrack]][VP[V$^{\prime}$[V[loved]][DP[$t_i$ ,name=object]]]]]]]]]]]
%%movement
\draw[->](object)to[out=south west ,in=south](spec IPB);
\end{forest}}
\end{enumerate}
(4)に見られるように, every man$_i$は最も近いIPに付加するので, LFにおいても明らかにhim$_i$をc統御しない. したがって, 非文であることが正しく予測される. 以上の議論から, QRの存在の妥当性が言えた.


\section{Exercise 5}
\begin{boxnote}
$\S$7.3において, 虚辞$\mathit{there}$はLFで連合要素(associate)に置き換えられると仮定した. そのため, 1はLFでは2のように表示される{$\colon$}
\begin{enumerate}
\item[]
\indent 1. There is a man in the room.\\
\indent 2. A man$_i$ is $t_i$ in the room.
\end{enumerate}
次の文はこの仮定に矛盾する. どのような問題があるか説明せよ.
\begin{enumerate}
\item[]
\indent 3. There weren't many books on the shelf.\\
\indent 4. Many books weren't on the shelf.
\end{enumerate}
\end{boxnote}
\vspace{10pt}
\noindent 3は「本棚にはあまり本がなかった」, 4は「多くの本が本棚になかった」ことを意味する. すなわち, 3は本の全体量が少ない, 例えば3冊しかないような状況を指すが, 4は極端な例を挙げれば, 本が10$^{10}$冊ある一方で, 探している本は1冊しかないような状況を指す. さて, 虚辞はLFで連合要素に置き換えられると仮定したので, 3の文のLFは次の様である$\colon$
\begin{enumerate}
\item[]
LF表示$\colon$
many books$_i$ weren't $t_i$ on the shelf.
\end{enumerate}
これは4に他ならない. このときうえの議論から, 3はS構造とLFで意味が変わってしまうことになる. 

\section{Exercise 6}
\begin{boxnote}
$\S$7.3では, 主語と定型動詞の一致(subject-verb agreement)が, D構造でなくLFで満たされていれば十分であることを見た. これは1のような文を説明するためであった{$\colon$}
\begin{enumerate}
\item[]
\indent 1. There were six people in the office.
\end{enumerate}
このように, 束縛原理(Binding Theory)が満たされるべき表示レベルもLFに修正する必要はあるか. 以下のような例を用いて説明せよ.
\begin{enumerate}
\item[]
\indent 2. John$_i$ bought every picture of himself$_i$\\
\indent 3. *He$_i$ liked every picture that John$_i$ took.
\end{enumerate}
ここまでの議論で得た結論は, 次の文に説明を与えるか.
\begin{enumerate}
\item[]
\indent 4. Which claim that John$_i$ made did he$_i$ deny?
\end{enumerate}
\end{boxnote}

\setcounter{subsection}{+1}
\subsection{}
\noindent John$_i$は指示表現(R-expression)なので束縛原理(binding principle)(C)が適用され, A自由(A-free)でなければならない. 他方, himself$_i$は照応形(anaphor)なので, 束縛原理(A)が適用され, その統率範疇(governing category)の内部で, A位置の要素に束縛(bind)されなければならない. 
\begin{enumerate}
\item[(1)]
S構造$\colon$
[{\tiny CP} [{\tiny IP} John$_i$ bought [{\tiny DP} every picture of himself$_i$]]].
\vspace{10pt}
\\
(1)
{\fontsize{9pt}{9pt}\selectfont
\begin{forest}
 %% triangle
[CP[C$^{\prime}$[C[$\emptyset$]][IP[DP[John$_i$ ,roof]][I$^{\prime}$[I[\lbrack Past\rbrack]][VP[V$^{\prime}$[V[bought]][DP[D$^{\prime}$[D[every]][NP[N$^{\prime}$[N[picture]][PP[P$^{\prime}$[P[of]][DP[himself$_i$ ,roof]]]]]]]]]]]]]]
\end{forest}}
\end{enumerate}
S構造においては, もちろん束縛原理を満たす. 実際, John$_i$の先行詞が存在せず, himself$_i$の統率範疇は主節のIPであり($\because$DP-himself$_i$, 統率子(governor)P-of, 接近可能な拡大主語(accessible SUBJECT)DP-John$_i$を含む最小のXPである), その中でJohn$_i$に束縛される. 
\newpage
\begin{enumerate}
\item[(2)]
LF表示$\colon$
[{\tiny CP} [{\tiny IPA} [{\tiny DP} every picture of himself$_i$]$_j$ [{\tiny IPB} John$_i$ bought $t_j$]]].
\vspace{10pt}
\\
(2)
{\fontsize{9pt}{9pt}\selectfont
\begin{forest}
 %% triangle
delay={where n children=0{if={instr("P",content("!u"))}{roof}{}}{}}
[IP$_A$[DP[\lbrack every picture of himself$_i$\rbrack$_j$]][IP$_B$[DP[John$_i$]][I$^{\prime}$[I[\lbrack Past\rbrack]][VP[bought $t_j$]]]]]
\end{forest}}
\end{enumerate}
しかしLFにおいては, 明らかにhimself$_i$がJohn$_i$にc統御されないので, 束縛原理(C)に違反する.

\subsection{}
\noindent he$_i$は代名詞なので束縛原理(B)が適用され, その統率範疇の内部で A自由でなければならない. John$_i$は指示表現なので束縛原理(C)が適用され, A自由でなければならない.
\begin{enumerate}
\item[(3)]
S構造$\colon$
*[{\tiny CP} [{\tiny IP} he$_i$ liked [{\tiny DP} every picture [{\tiny CP} Op$_j$ that John$_i$ took $t_j$]]]].
\vspace{10pt}
\\
(3)
{\fontsize{9pt}{9pt}\selectfont
\begin{forest}
 %% triangle
delay={where n children=0{if={instr("P",content("!u"))}{roof}{}}{}}
[IP[DP[he$_i$]][I$^{\prime}$[I[\lbrack Past\rbrack]][VP[liked every picture Op$_j$ that John$_i$ took $t_j$]]]]
\end{forest}}
\end{enumerate}
S構造において束縛原理に違反する. 実際, A位置にあるhe$_i$がJohn$_i$を束縛し, John$_i$はA自由でない. 
\newpage
\begin{enumerate}
\item[(4)]
LF表示$\colon$
[{tiny CP} [{\tiny IPA} [{\tiny DP} every picture Op$_j$ that John$_i$ took $t_j$]$_k$ [{\tiny IPB} he$_i$ liked $t_k$]]].
\vspace{10pt}
\\
(4)
{\fontsize{9pt}{9pt}\selectfont
\begin{forest}
 %% triangle
delay={where n children=0{if={instr("P",content("!u"))}{roof}{}}{}}
[IPA[DP$_k$[D$^{\prime}$[D[every]][NP[N$^{\prime}$[N[picture]][CP[DP[Op$_j$]][C$^{\prime}$[C[that]][IP[DP[John$_i$]][I$^{\prime}$[I[\lbrack Past\rbrack]][VP[V$^{\prime}$[V[took]][DP[$t_j$]]]]]]]]]]]][IPB[DP[he$_i$]][I${\prime}$[I[\lbrack Past\rbrack]][VP[V$^{\prime}$[V[liked]][DP[$t_k$]]]]]]]
\end{forest}}
\end{enumerate}
LFにおいて束縛原理を満たす. 実際, he$_i$, John$_i$は互いにc統御せず, すなわちともにA自由である.\\
以上の議論から, LFで束縛原理を満たせばよいとすれば, 2を非文予測し, 3を正文予測してしまう. したがって, これまでと同様, S構造で束縛原理を確かめればよい.
\subsection{}
\noindent he$_i$は代名詞なので束縛原理(B)が適用され, その統率範疇の内部で A自由でなければならない. John$_i$は指示表現なので束縛原理(C)が適用され, A自由でなければならない.
\begin{enumerate}
\item[(5)]
D構造$\colon$
*[{\tiny CP} [{\tiny IP} he$_i$ deny [{\tiny DP} which claim [{\tiny CP} Op$_j$ that John$_i$ made $t_j$]]]].
\vspace{10pt}
\\
(5)
{\fontsize{9pt}{9pt}\selectfont
\begin{forest}
 %% triangle
delay={where n children=0{if={instr("P",content("!u"))}{roof}{}}{}}
[IP[DP[he$_i$]][I$^{\prime}$[I[\lbrack Past\rbrack]][VP[deny which Op$_j$ that John$_i$ made $t_j$]]]]
\end{forest}}
\end{enumerate}
D構造において束縛原理に違反する. 実際, A位置にあるhe$_i$がJohn$_i$を束縛し, John$_i$はA自由でない. 
\begin{enumerate}
\item[(6)]
S構造$\colon$
[{\tiny CP} [{\tiny DP} Which claim Op$_j$ that John$_i$ made $t_j$]$_k$ did [{\tiny IP} he$_i$ deny $t_k$]]
\vspace{10pt}
\\
(6)
{\fontsize{9pt}{9pt}\selectfont
\begin{forest}
 %% triangle
delay={where n children=0{if={instr("P",content("!u"))}{roof}{}}{}}
[CP[DP$_k$[D$^{\prime}$[D[which]][NP[N$^{\prime}$[N[claim]][CP[DP[Op$_j$]][C$^{\prime}$[C[that]][IP[DP[John$_i$]][I$^{\prime}$[I[\lbrack Past\rbrack]][VP[V$^{\prime}$[V[made]][DP[$t_j$]]]]]]]]]]]][C$^{\prime}$[C[did]][IP[DP[he$_i$]][I$^{\prime}$[I[\lbrack Past\rbrack]][VP[V[deny]][DP[$t_k$]]]]]]]
\end{forest}}
\end{enumerate}
S構造において束縛原理を満たす. 実際, he$_i$, John$_i$は互いにc統御せず, すなわちともにA自由である. ここに(3)と(5), (4)と(6)の対応を見ることができる. よって, S構造で束縛原理を満たすべきという結論が従う.





\end{document}

